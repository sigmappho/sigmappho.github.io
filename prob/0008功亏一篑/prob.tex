\documentclass{article}
\usepackage{ctex}
\usepackage{amsmath}
\usepackage{graphicx}
\usepackage{wrapfig}
\usepackage{caption}
\usepackage[top=0.8in, bottom=0.8in,left=0.8in, right=0.8in]{geometry}
\usepackage{float} 
\usepackage{subfigure}
\usepackage{subcaption}
\usepackage{bm}
\xeCJKsetup{CJKmath=true} 

\begin{document}
\section*{功亏一篑(40分)}
众所周知,台球先打彩球,最后打黑8,但若打黑8时白球同时进洞,则直接判对方获胜。小s同学就经常干这类愚蠢之事,现建模分析。\par
若已知每个球的质量为$m$,重力加速度为$g$,桌面滑动摩擦系数为$\mu$,半径为$R$。
\begin{itemize}
\item[(1)]求打击何处时,可以使其纯滚。
\item[(2)]若小s同用(1)方式击打白球,使其获得速度$v_0$,撞击与洞口距离为l的黑8(弹性正撞,且白球,黑8,洞口共线)。
\begin{itemize}
\item[(2.1)]白球是否会进洞?
\item[(2.2)]求白球在入洞前相对桌面的滑动距离。
\item[(2.3)]若白球进洞,求白球在黑球之后多久入洞。若白球不如洞,求白球停在距洞多远处。
\end{itemize}
\end{itemize}
\end{document} 