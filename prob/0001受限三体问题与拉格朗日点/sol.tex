\documentclass{article}
\usepackage{ctex}
\usepackage{amsmath}
\usepackage{graphicx}
\usepackage{wrapfig}
\usepackage{caption}
\usepackage[top=0.8in, bottom=0.8in,left=0.8in, right=0.8in]{geometry}
\usepackage{float} 
\usepackage{subfigure}
\usepackage{subcaption}
\usepackage{bm}
\xeCJKsetup{CJKmath=true} 

\begin{document}
\section*{受限三体问题与拉格朗日点(60分)(命题:HJH)}
\[\]
(1)
先求公转角速度
\[
\dfrac{Gm_1 m_2}{r^2}=\Omega^2 r_1 m_1=\Omega^2 \dfrac{m_1 m_2 r}{m_1 +m_2}
\]
\[
=>\Omega^2=\dfrac{G(m_1+m_2)}{r^3}\tag{1}
\]
\[
V_{\mathbf{eff}}(x,y)=
\dfrac{-Gm_1}{\sqrt{(x-r_1)^2+y^2}}
-\dfrac{Gm_2}{(\sqrt{(x+r_2)^2+y^2}}
-\dfrac{1}{2}\dfrac{G(m_1+m_2)}{r^3}(x^2+y^2)\tag{2}
\]
(2)
\[
\dfrac{\partial V_{\mathbf{eff}}}{\partial x}
=0
=\dfrac{Gm_1(x-r_1)}{[(x-r_1)^2+y^{2}]^{\frac{3}{2}}}
+\dfrac{Gm_2(x+r_2)}{[(x+r_2)^2+y^{2}]^{\frac{3}{2}}}
-\dfrac{G(m_1+m_2)}{r^3}x
\tag{3}
\]
\[
\dfrac{\partial V_{\mathbf{eff}}}{\partial y}=0
=\dfrac{Gm_1 y}{[(x-r_1)^2+y^{2}]^{\frac{3}{2}}}
+\dfrac{Gm_2 y}{[(x+r_2)^2+y^{2}]^{\frac{3}{2}}}
-\dfrac{G(m_1+m_2)}{r^3}y
\tag{4}
\]
对于$y=0,(4)$易知满足,对于$(3)$,同除$\dfrac{G(m_1+m_2)}{r}$,得
\[
 \dfrac{r_2(x-r_1)}{|x-r_1|^3}
+\dfrac{r_2(x+r_2)}{|x+r_2|^3}
=\dfrac{x}{r^2}\tag{3'}
\]
令$f(x)=\dfrac{r_2(x-r_1)}{|x-r_1|^3}+\dfrac{r_2(x+r_2)}{|x+r_2|^3}$,有
\[
\lim_{x \to -\infty}f(x)=0^-\tag{5}
\]
\[
\lim_{x \to (-r_2)^-}f(x)=-\infty,\lim_{x \to (-r_2)^+}f(x)=+\infty\tag{6}
\]
\[
\lim_{x \to (r_1)^-}f(x)=-\infty,\lim_{x \to (r_1)^+}f(x)=+\infty\tag{7}
\]
\[
\lim_{x \to +\infty}f(x)=0^+\tag{8}
\]
\[可见,有3根\tag{9}\]
\[L_1(-\infty,-r_2),L_2(-r_2,r_1),L_3(r_1,\infty)\tag{10}\]
(3)
在$\dfrac{m_2}{m_1}=\varepsilon\to 0$
\[
r_1=\dfrac{m_2}{m_1+m_2}r=\dfrac{\varepsilon}{\varepsilon+1}r\approx \varepsilon(1-\varepsilon)r\tag{11}
\]
\[
r_2=\dfrac{m_1}{m_1+m_2}r=\dfrac{1}{\varepsilon+1}r\approx (1-\varepsilon)r\tag{12}
\]
保留零阶,代入$(4)$式
\[
\dfrac{rx}{|x|^3}=\dfrac{x}{r^2}\tag{13}
\]
\[
=>x=\pm r\tag{14}
\]
\[
可得L_1,L_2在m_2附近,L_3在离原点r附近\tag{15}
\]
(4)
对于$L_1,L_2$,令$x=-r+\delta_{1,2}r$,带入$(4)$式,保留一阶
\[
-\dfrac{(1-\varepsilon)r}{(-r+\delta_{1,2}r-\varepsilon r)^2}
\pm 
\dfrac{\varepsilon r}{(-r + \delta_{1,2} r +(1-\varepsilon)r)^2}=\dfrac{-r+\delta_{1,2}r}{r^2}\tag{16}
\]
\[
\pm \dfrac{\varepsilon}{(\delta_{1,2}-\varepsilon)^2}=3(\delta_{1,2}-\varepsilon)
\]
由于两边小量阶数一致,取$1\gg\delta_{1,2}\gg\varepsilon$ ,带入有
\[
\pm \dfrac{\varepsilon}{\delta_{1,2}^2}\left(1+\dfrac{2\varepsilon}{\delta_{1,2}}\right)=3(\delta_{1,2}-\varepsilon)
\]
\[
=>\pm  \dfrac{\varepsilon}{\delta_{1,2}^2}=3 \delta_{1,2}^2
\]
\[
=>\delta_{1,2}= \pm \left(\dfrac{\varepsilon}{3}\right)^{\frac{1}{3}}\tag{17}
\]
对于$L_3$,令$x=r+\delta_3$
\[
\dfrac{1-\varepsilon}{(1+\delta_3-\varepsilon)^2}+\dfrac{\varepsilon}{(1+\delta_3+1-\varepsilon)^2}=1+\delta_3\tag{18}
\]
得
\[
\delta_3=\dfrac{5}{12}\varepsilon\tag{19}
\]
故
\[
L_1\left(-r-(\dfrac{\varepsilon}{3})^{\frac{1}{3}}r,0\right)\tag{20}
\]
\[
L_2\left(-r+(\dfrac{\varepsilon}{3})^{\frac{1}{3}}r,0\right)\tag{21}
\]
\[
L_3\left(r+\dfrac{5}{12}\varepsilon r,0\right)\tag{22}
\]
(5)
由$(4)\times \dfrac{x}{y}$与$(3)$相减得
\[
\dfrac{-r_1 r_2}{[(r_1-x)^2+y^2]^{\frac{2}{3}}}
+\dfrac{r_1 r_2}{[(r_2+x)^2+y^2]^{\frac{2}{3}}}
=0\tag{23}
\]
由$(4)\times \dfrac{1}{y}$得
\[
\dfrac{r_2}{[(r_1-x)^2+y^2]^{\frac{2}{3}}}
+\dfrac{r_1}{[(r_2+x)^2+y^2]^{\frac{2}{3}}}
=\dfrac{1}{r^2}\tag{24}
\]
$(23),(24)$可看作关于分母的一元二次方程组,解得
\[
(r_1-x)^2+y^2=(r_2+x)^2+y^2=r^2\tag{25}
\]
\[
x=\dfrac{r_1-r_2}{2},y=\pm\dfrac{\sqrt{3}r}{2}\tag{26}
\]
\[
可得L_4,L_5与 m_1,m_2构成两个等边三角形\tag{27}
\]
\textbf{评分标准:}\par
共$60$ 分\par
(1)共$4$分 $(1),(2)$各$2$分\par
(2)共$16$分 $(3),(4)$各$2$分,$(5),(6),(7),(8)$各$2$分 $(9)$各$2$ $(10)$$2$分\par
(3)共$8$分 $(11),(12),(13)$各$1$分 $(14)$$3$分 $(15)$ $2$分\par
(4)共$20$分 $(16)$$2$分 $(17)$ $6$分 $(18)$$2$分 $(19)$各$4$分 $(20),(21),(22)$各$2$分\par
(5)共$12$分 $(23)$ $2$分 $(25)$ $4$分 $(26)$ $4$分 $(27)$ $2$分\par

\end{document} 