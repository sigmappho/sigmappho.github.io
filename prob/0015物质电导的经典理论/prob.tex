\documentclass{article}
\usepackage{ctex}
\usepackage{amsmath}
\usepackage{graphicx}
\usepackage{wrapfig}
\usepackage{caption}
\usepackage[top=0.8in, bottom=0.8in,left=0.8in, right=0.8in]{geometry}
\usepackage{float} 
\usepackage{subfigure}
\usepackage{subcaption}
\usepackage{bm}
\xeCJKsetup{CJKmath=true} 

\begin{document}
\section*{物质电导的经典理论(40分)}
经典电子论的基础是由P.K.L.Drude在1900左右提出的。其模型如下:
\begin{itemize}
\item[1.]将金属分为固定不动(可在附近做振动)的原子核和自由移动的电子(满足能均分定理)。
\item[2.]自由电子的运动决定了金属的导热性与导电性。
\item[3.]自由电子与原子核碰撞来交换能量,从而达到热平衡。
\end{itemize}
\begin{itemize}
\item[(1)]Ohm定律,简单认为电子以平均速率$\bar{u}$运动,且与原子核相碰后完全失去定向移动速率,设平均自由程为$\bar{\lambda}$,电子热运动以热运动平均速率$\bar{u}$运动。\par
试证明:
\[
\vec{j}=\sigma\vec{E}
\]
并给出$\sigma$,用电子质量$m$,电量绝对值$e$,数密度$n$(一价金属),平均自由程$\bar{\lambda}$表示。
\item[(2)]Joule-Lenz定律,电子与原子核相碰后,其动能完全转化为原子核的热振动动能,给出热运动功率密度(单位体积内放出的热能)。用电场$\vec{E}$与一个用上已知量表示的常数给出。
\item[(3)]Wiedemann-Frantz定律,在这里,我们将自由电子看成自由电子气,满足能均分定理,给出导热系数$\kappa$的微观表达式以及与电导率$\sigma$之间的关系
\end{itemize}
\par
提示:傅里叶热传导定律
\[
j_q=-\kappa\dfrac{\mathrm{d}T}{\mathrm{d}z}
\]
其中$j_q$为单位时间,流过单位面积的能量。\par
注意:本题无需考虑Maxwell速率分布
\end{document} 