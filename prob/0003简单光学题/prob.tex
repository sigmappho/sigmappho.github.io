\documentclass{article}
\usepackage{ctex}
\usepackage{amsmath}
\usepackage{graphicx}
\usepackage{wrapfig}
\usepackage{caption}
\usepackage[top=0.8in, bottom=0.8in,left=0.8in, right=0.8in]{geometry}
\usepackage{float} 
\usepackage{subfigure}
\usepackage{subcaption}
\usepackage{bm}
\xeCJKsetup{CJKmath=true} 
\begin{document}
\section*{简单光学题(40分)}
\begin{itemize}
\item[(1)]一宽平行光束正入射到折射率为$n$的平凸透镜左侧平面,汇聚到平凸透镜主轴上的$F$点,已知$\overline{OF}=r_0$给出凸面型状,并给出其在直角坐标系下的标准方程,(需声明原点)。
\item[(2)]磁场透镜\par
一宽束质量,速度,电荷量分别为$m,v,q$入射到$x=0$平面。已知在第一、四象限存在大小为$B$,方向相反的磁场区域,出射后汇聚于$F(f,0)$处,给出磁场区域边界方程。
\item[(3)]电场透镜\par
一宽束质量,速度,电荷量分别为$m,v,q$入射到$x=0$平面。全空间中分布着如下电场
\[
\vec{E}=
\begin{cases}
-ky^{\alpha}\hat{y}(y>0)\\
ky^{\alpha}\hat{y}(y<0)
\end{cases}
\]
出射后汇聚于$F(f,0)$处,通过一些性质给出$\alpha$,并给出$k$的定量表达式。
\item[(4)]正经光学题\par
一束光平行于$x$轴方向入射,第一、四象限存在折射率只与$y$有关的介质,其边界是锯齿状的,使得光能垂直接着入射,出射后汇聚于$F(f,0)$处,在$(0,0)$处折射率为$n_0$,类比$(3)$给出折射率分布与边界方程.
\end{itemize}
\end{document}