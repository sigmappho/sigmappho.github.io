\documentclass{article}
\usepackage{ctex}
\usepackage{amsmath}
\usepackage{graphicx}
\usepackage{wrapfig}
\usepackage{caption}
\usepackage[top=0.8in, bottom=0.8in,left=0.8in, right=0.8in]{geometry}
\usepackage{float} 
\usepackage{subfigure}
\usepackage{subcaption}
\usepackage{bm}
\xeCJKsetup{CJKmath=true} 
\begin{document}
\section*{打飞机奇遇(40分)}
小s同学一天看到天上飞过一架漂亮国的飞机,于是想设计一个电磁炮把它打下,现在来分析这一过程。
\begin{itemize}
\item[(1)]小s同学先设计了一个可以产生$\vec{B}=k\cdot z^{\frac{3}{2}}\hat{z}$磁感应强度的线圈,然后以$v_0$从$z=0$(以炮弹尾而言)处发射一枚炮声电导率为$\sigma$的金属圆柱壳,其厚度为$t$,弹头为绝缘材料的炮弹,已知炮管长$L$,不计重力,求出射时炮弹速度。炮身长$l$,半径为$r$,炮弹总质量为$m$.
\item[(2)]但是小s同学惊奇的发现(1)中的炮弹出射速度小于$v_0$,于是他重新进行设计,仅将电导率为$\sigma$的金属壳换位超导圆柱壳,初始磁通量为零,磁感应强度改为$\vec{B}=(B_0-kz)\hat{z}$,其他参量均不变,且$B_0>k(L+\frac{l}{2})$将炮弹从$z=0$处静止释放,不记重力,求出射速度。
\item[(3)]设计完成后他还不满足,又设计了一新型炮弹,全部为绝缘材料,但体心有一微小的,通有恒流$I$,半径为$R$的金属环(可视为磁偶极子)磁场改为$\vec{B}=k\cdot z^{\alpha}\hat{z}$,炮弹质量为$m$,炮身长$L$,初始时位于$z=0$处,求出射速度。
\end{itemize}
\end{document} 