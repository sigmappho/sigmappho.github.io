\documentclass{article}
\usepackage{ctex}
\usepackage{amsmath}
\usepackage{graphicx}
\usepackage{wrapfig}
\usepackage{caption}
\usepackage[top=0.8in, bottom=0.8in,left=0.8in, right=0.8in]{geometry}
\usepackage{float} 
\usepackage{subfigure}
\usepackage{subcaption}
\usepackage{bm}
\usepackage{tikz}
\usetikzlibrary{arrows.meta}
% \xeCJKsetup{CJKmath=true} 
\begin{document}
% \everymath{\displaystyle}
\section*{van der Waals相互作用}
\begin{itemize}
    \item[(1)]由偶极矩在外场中的电势能公式
    \begin{equation}
        U=-\overrightarrow{p}\cdot\overrightarrow{E}
    \end{equation}
    以及电偶极子的电场力公式
    \begin{equation}
        \overrightarrow{E}=\dfrac{1}{4\pi\varepsilon_0}\dfrac{3(\overrightarrow{p}\cdot\hat{r})\hat{r}-\overrightarrow{p}}{r^3}
    \end{equation}
    \begin{center}
        \begin{tikzpicture}
            \draw (-6,0) -- (6,0);
            \draw[-{Stealth[length=4mm,width=2mm]}] (-5,-1) -- (-3,1);
            \draw[-{Stealth[length=4mm,width=2mm]}] (2.5,-1)-- (5.5,1);
            \filldraw [black]
            (-5,-1) circle [radius=2pt]
            (-3,1) circle [radius=2pt]
            (2.5,-1) circle [radius=2pt]
            (5.5,1) circle [radius=2pt];
            \draw (-3.7,0) arc [start angle=0, end angle=45, radius=3mm];
            \draw (4.3,0) arc [start angle=0, end angle=33.8, radius=3mm];
            \draw (-3.2,0) node[above] {$\theta_1$};
            \draw (4.8,0) node[above] {$\theta_2$};
            \draw (-4,0) node[above] {$\overrightarrow{p_1}$};
            \draw (4,0) node[above] {$\overrightarrow{p_2}$};
            \draw [<-](2,0) arc (0:300:0.5 and 1);
            \draw (2,1) node[above] {相对转动角$\varphi$};
            \draw (0,0) node[above] {$\overrightarrow{r}$};
        \end{tikzpicture}
    \end{center}
    可得其电势能大小为
    \begin{equation}
    \begin{aligned}
        U(\theta_1,\theta_2,\varphi,r)   &= -\overrightarrow{p}\cdot\overrightarrow{E}\\
            &= -\dfrac{1}{4\pi\varepsilon_0}\dfrac{3(\overrightarrow{p_1}\cdot\hat{r})\hat{r}-\overrightarrow{p_1}}{r^3}\cdot\overrightarrow{p_2}\\
            &= -\dfrac{1}{4\pi\varepsilon_0}\dfrac{3(\overrightarrow{p_1}\cdot\hat{r})\hat{r}\cdot\overrightarrow{p_2}-\overrightarrow{p_1}\cdot\overrightarrow{p_2}}{r^3}\\
            &= -\dfrac{1}{4\pi\varepsilon_0}\dfrac{3p_1p_2\cos\theta_1\cos\theta_2-p_1p_2(\cos\theta_1\cos\theta_2+\sin\theta_1\sin\theta_2\cos\varphi)}{r^3}\\
            &= -\dfrac{1}{4\pi\varepsilon_0}\dfrac{2p_1p_2\cos\theta_1\cos\theta_2-p_1p_2\sin\theta_1\sin\theta_2\cos\varphi}{r^3}\\
    \end{aligned}
    \end{equation}
    \item[(2)]按照定义
    \begin{equation}
        \overline{U}=\dfrac{\displaystyle\int_1\displaystyle\int_2U(\theta_1,\theta_2,\varphi,r)\exp\left(-\dfrac{U(\theta_1,\theta_2,\varphi,r)}{k_B T}\right)\mathrm{d}\Omega_1\mathrm{d}\Omega_2}{\displaystyle\int_1\displaystyle\int_2\exp\left(-\dfrac{U(\theta_1,\theta_2,\varphi,r)}{k_B T}\right)\mathrm{d}\Omega_1\mathrm{d}\Omega_2}
    \end{equation}
    令
    \begin{equation}
        \begin{aligned}
            \beta=&\dfrac{1}{4\pi\varepsilon_0}\dfrac{p_1p_2}{r^3}\dfrac{1}{k_B T}\\
            f(\Omega)=&2\cos\theta_1\cos\theta_2-\sin\theta_1\sin\theta_2\cos\varphi
        \end{aligned}
    \end{equation}
    上式化为
    \begin{equation}
        \begin{aligned}
        \overline{U}&=\dfrac{\displaystyle\int_1\displaystyle\int_2 k_B T\beta f(\Omega)\exp\left(\beta f(\Omega)\right)\mathrm{d}\Omega_1\mathrm{d}\Omega_2}{\displaystyle\int_1\displaystyle\int_2\exp\left(\beta f(\Omega)\right)\mathrm{d}\Omega_1\mathrm{d}\Omega_2}\\    
        &= k_B T\beta\dfrac{-\dfrac{\partial}{\partial \beta}\displaystyle\int_1\displaystyle\int_2\exp\left(\beta f(\Omega)\right)\mathrm{d}\Omega_1\mathrm{d}\Omega_2}{\displaystyle\int_1\displaystyle\int_2\exp\left(\beta f(\Omega)\right)\mathrm{d}\Omega_1\mathrm{d}\Omega_2}\\
        &= -k_B T\beta\dfrac{\partial}{\partial \beta}\ln\left(\displaystyle\int_1\displaystyle\int_2\exp\left(\beta f(\Omega)\right)\mathrm{d}\Omega_1\mathrm{d}\Omega_2\right)\\
        \end{aligned}
    \end{equation}
    易知
    \begin{equation}
        \mathrm{d}\Omega_1=\sin\theta_1 \mathrm{d}\theta_1 \mathrm{d} \varphi_1
    \end{equation}
    \begin{equation}
        \mathrm{d}\Omega_2=\sin\theta_2 \mathrm{d}\theta_2 \mathrm{d} \varphi_2
    \end{equation}
    而由于引入了相对转动角,故
    \begin{equation}
        \mathrm{d}\Omega_1\mathrm{d}\Omega_2=\sin\theta_1 \mathrm{d}\theta_1\sin\theta_2 \mathrm{d}\theta_2 \mathrm{d} \varphi
    \end{equation}
    下求积分
    \begin{equation}
        I=
        \displaystyle\int_{\theta_1=0}^{\pi} \displaystyle\int_{\theta_2=0}^{\pi}\displaystyle\int_{\varphi=0}^{2\pi}
        \exp(\beta f(\Omega))\sin \theta_1 \mathrm{d} \theta_1\sin \theta_2 \mathrm{d} \theta_2  \mathrm{d} \varphi
    \end{equation}   
    代入题给近似
    \begin{equation}
        U(\theta_1,\theta_2,\varphi,r) \sim \dfrac{1}{4\pi\varepsilon_0}\dfrac{\overrightarrow{p_1}\cdot\overrightarrow{p_2}}{r^3}\ll k_B T
    \end{equation}
    与Tailor展开公式
    \begin{equation}
        \exp(x)\approx 1+x+\dfrac{1}{2}x^2\qquad x\ll1
    \end{equation}
    \begin{equation}
        \begin{aligned}
        I\approx&
        \displaystyle\int_{\theta_1=0}^{\pi} \displaystyle\int_{\theta_2=0}^{\pi}\displaystyle\int_{\varphi=0}^{2\pi}
        \left(1+\beta f(\Omega)+\dfrac{1}{2}\left(\beta f(\Omega)\right)^2\right)\sin \theta_1 \mathrm{d} \theta_1\sin \theta_2 \mathrm{d} \theta_2  \mathrm{d} \varphi\\
        =&I_0+I_1+I_2            
        \end{aligned}
    \end{equation}  
    其中$I_0,I_1,I_2$分别对应其零阶,一阶,二阶近似的积分结果。易知
    \begin{equation}
        I_0=8\pi
    \end{equation}
    \begin{equation}
        I_1=0
    \end{equation}
    \begin{equation}
        I_2=\dfrac{8\pi}{3}\beta^2
    \end{equation}
    故有
    \begin{equation}
        \begin{aligned}
        \overline{U}&\approx-k_B T\beta\dfrac{\partial}{\partial \beta}\ln\left(8\pi(1+\dfrac{\beta^2}{3})\right)\\
            &=-k_B T\beta\dfrac{1}{1+\dfrac{\beta^2}{3}}\cdot\dfrac{2\beta}{3}\\
            &\approx -k_B T\dfrac{2\beta^2}{3}\\
            &=-\dfrac{2}{3k_B T}\left(\dfrac{p_1p_2}{4\pi\varepsilon_0}\right)^2\cdot\dfrac{1}{r^6}
        \end{aligned}
    \end{equation}
    \item[(3)]在$p_2$处,电场强度大小为
    \begin{equation}
        |\overrightarrow{E}|=\dfrac{p_1}{4\pi\varepsilon_0}\dfrac{\sqrt{1+3\cos^2\theta_1}}{r^3}
    \end{equation}
    故其偶极矩大小为
    \begin{equation}
        |\overrightarrow{p}|=\alpha\dfrac{p_1}{4\pi\varepsilon_0}\dfrac{\sqrt{1+3\cos^2\theta_1}}{r^3}
    \end{equation}    
    代入(1)式
    \begin{equation}
        U=-\alpha\dfrac{p_1^2}{(4\pi\varepsilon_0)^2}\dfrac{1+3\cos^2\theta_1}{r^6}
    \end{equation}   
    \item[(4)]此时,有
    \begin{equation}
        \overline{U}=\dfrac{\displaystyle\int_{\theta_1=0}^{\pi}U\exp\left(-\dfrac{U}{k_B T}\right)\sin\theta_1\mathrm{d}\theta_1}{\displaystyle\int_{\theta_1=0}^{\pi}\exp\left(-\dfrac{U}{k_B T}\right)\sin\theta_1\mathrm{d}\theta_1}
    \end{equation}
    保留至零阶项
    \begin{equation}
        \overline{U}=-\alpha\dfrac{2p_1^2}{(4\pi\varepsilon_0)^2r^6}
    \end{equation}
\end{itemize}
\end{document} 