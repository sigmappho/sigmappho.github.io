\documentclass{article}
\usepackage{ctex}
\usepackage{amsmath}
\usepackage{graphicx}
\usepackage{wrapfig}
\usepackage{caption}
\usepackage[top=0.8in, bottom=0.8in,left=0.8in, right=0.8in]{geometry}
\usepackage{float} 
\usepackage{subfigure}
\usepackage{subcaption}
\usepackage{bm}
\xeCJKsetup{CJKmath=true} 

\begin{document}
\section*{受限三体问题与拉格朗日点(60分)}
	对于任意给定的$m_1,m_2,m_3$,仅在万有引力的作用下运动,在任意给定初值的条件下求解$m_1,m_2,m_3$的运动的问题称作三体问题,时至今日依旧没有解析解,但对于$m_1,m_2\gg m$的情况下,且完全忽略$m$对$m_1,m_2$运动的影响,称为受限三体问题。\par
	在这类问题中,有一些点满足在$m_1,m_2$公转系中静止的条件,这些点称为拉格朗日点。\par
	下认为$m_1,m_2$均作圆周运动,以质心为原点,$(r_1,0)$表示$m_1$的位置,$(-r_2,0)$表示$m_2$的位置,且$r=r_1+r_2$。\par
\begin{itemize}
\item[(1)]	给出任意$(x,y)$处的有效势.(单位质量势能,$m_1,m_2$除外)
\item[(2)]	给出拉格朗日点满足的方程(无需求解)给出$y=0$拉格朗日点的个数,并定性描述其位置.\par
在一定近似下,我们可以求解,如令
$$\varepsilon=\dfrac{m_1}{m_2}\to 0.$$
\item[(3)]	给出$y=0$时的零阶解,并进一步描述位置.
\item[(4)]	给出$y=0$时的一阶解,并给出坐标.
\item[(5)]	求出剩下的点,并指出其特殊几何关系.
\end{itemize}
\end{document} 