\documentclass{article}
\usepackage{ctex}
\usepackage{amsmath}
\usepackage{graphicx}
\usepackage{wrapfig}
\usepackage{caption}
\usepackage[top=0.8in, bottom=0.8in,left=0.8in, right=0.8in]{geometry}
\usepackage{float} 
\usepackage{subfigure}
\usepackage{subcaption}
\usepackage{bm}
\xeCJKsetup{CJKmath=true} 
\newcommand \dbar {{\; \bar{} \hspace{-0.3em} \mathrm d}}
% \newcommand \d {{\; \bar{}  \mathrm d}}
\begin{document}
\section*{液体表面张力系数的确定(40分)(命题人:HJH)}
\[\]
(1)由表面张力公式
\[F=l\sigma\tag{1}\]
表面张力作功
\[\dbar W=F\cdot \mathrm{d}x=\sigma\mathrm{d}A\tag{2}\]
\[\mathrm{d}A=l\mathrm{d}x\tag{3}\]
代入可得
\[\mathrm{d}U=\sigma\mathrm{d}S+T\mathrm{d}S\tag{4}\]
(2)由题意
\[
\Delta U=(1-\zeta)\dfrac{-\varepsilon}{2}n
\tag{5}\]
(3)由题意
\[
\dbar W=\sigma_n\mathrm{d}A_S(1-\zeta)\dfrac{- \varepsilon}{2}n
\tag{6}\]
(4)由题意
\[
-\dfrac{n}{2}\varepsilon N_A=L_m
\tag{7}\]
\[
\varepsilon=\dfrac{-2L_m}{nN_A}
\tag{8}\]
(5)
\[
d^3N_A=\dfrac{\mu}{\rho}
\tag{9}\]
\[
d=\left(\dfrac{\mu}{\rho N_A}\right)^{\frac{1}{3}}
\tag{10}\]
\[
\sigma=(1-\zeta)\dfrac{L_m}{N_A}\left(\dfrac{\rho N_A}{\mu}\right)^{\frac{2}{3}}
\tag{11}\]
(6)类似于上面的方法
\[
\Delta U_1=(1-\zeta_1)x_1\dfrac{n_1}{n_1+n_2}\dfrac{-\varepsilon_{11}}{2}+(1-\zeta_1)x_1\dfrac{n_2}{n_1+n_2}\dfrac{-\varepsilon_{12}}{2}
\tag{12}\]
\[
\Delta U_2=(1-\zeta_2)x_1\dfrac{n_1}{n_1+n_2}\dfrac{-\varepsilon_{12}}{2}+(1-\zeta_1)x_1\dfrac{n_2}{n_1+n_2}\dfrac{-\varepsilon_{22}}{2}
\tag{13}\]
\[
\dbar W=\sigma_{n_1}\mathrm{d} A_S \Delta U_1+\sigma_{n_2}\mathrm{d} A_S \Delta U_2=\sigma_{12}\mathrm{d} A_S
\tag{14}\]
\[
\begin{aligned}
\sigma_{12}=\left[(1-\zeta_1)x_1\dfrac{n_1}{n_1+n_2}\dfrac{-\varepsilon_{11}}{2}+(1-\zeta_1)x_1\dfrac{n_2}{n_1+n_2}\dfrac{-\sqrt{\varepsilon_{11}\varepsilon_{22}}}{2}\right]\sigma_{n_1}\\
+\left[(1-\zeta_2)x_1\dfrac{n_1}{n_1+n_2}\dfrac{-\sqrt{\varepsilon_{11}\varepsilon_{22}}}{2}+(1-\zeta_1)x_1\dfrac{n_2}{n_1+n_2}\dfrac{-\varepsilon_{22}}{2}\right]\sigma_{n_2}
\end{aligned}
\tag{15}\]

\textbf{评分标准:}\par
共$40$ 分\par
(1)共$7$分 $(1),(3)$各$1$分,$(2)$$2$分,$(4)$$3$分\par
(2)共$3$分 $(5)$$3$分\par
(3)共$3$分 $(6)$$3$分\par
(4)共$6$分 $(7),(8)$各$3$分\par
(5)共$9$分 $(9),(10),(11)$各$3$分\par
(5)共$12$分$(12),(13),(14),(15)$各$4$分
\end{document} 