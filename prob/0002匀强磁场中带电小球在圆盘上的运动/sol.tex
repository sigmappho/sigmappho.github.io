\documentclass{article}
\usepackage{ctex}
\usepackage{amsmath}
\usepackage{graphicx}
\usepackage{wrapfig}
\usepackage{caption}
\usepackage[top=0.8in, bottom=0.8in,left=0.8in, right=0.8in]{geometry}
\usepackage{float} 
\usepackage{subfigure}
\usepackage{subcaption}
\usepackage{bm}
\xeCJKsetup{CJKmath=true} 
\begin{document}
\section*{匀强磁场中带电小球在圆盘上的运动(50分)(命题:LJT)}
\[\]
(1)解:按定义
\[\vec{\mu }=\dfrac{1}{2}\int \rho \vec{r}\times \vec{v}dV\tag{1}\]
\[Q=\dfrac{4}{3}\pi R^{3}\rho\tag{2}\]
得
\[\vec{\mu }=\dfrac{1}{5}Q\vec{w}r^{2}\tag{3}\]
(2)解:由无相对滑动,故$\omega_z=0$
\[\vec{v_{c}}+\left( -r\hat{z}\right) \times \vec{w}=\left( x\hat{x}+y\hat{y}\right) \times \left( \Omega \hat{z}\right) \tag{4}\]
\[
\begin{cases}
    \dot{x}-\omega_{y}r=-\Omega y\\
    \dot{y}+\omega_{x} r=\Omega x
\end{cases}
\tag{5}
\]
质心运动定理
\[
F_x \hat{x}+F_y\hat{y}+Q\vec{v}_{c}\times \left( B\hat{z}\right) =m\left( \ddot{x}\hat{x}+ \ddot{y}\hat{y}\right)
\tag{7}
\]
质心转动定理
\[
\dfrac{2}{5}mr^{2}\dfrac{\mathrm{d}\vec{\omega}}{\mathrm{d}t}=\left( -r\right) \hat{z}\times \left( F_{x}\hat{x}+F_{y}\hat{y}\right) +\vec{\mu }\times \vec{B}
\tag{8}
\]
\[
\begin{cases}
    rF_y+\mu _{y}B=\dfrac{2}{5}mr^{2}\dot{\omega_x}\\
    -\left( rF_{x}+\mu _{x}B\right) =\dfrac{2}{5}mr^{2}\dot{\omega_y}
\end{cases}
\tag{9}
\]
消去$F_z,F_y,\omega_x,\omega_y$,得二阶二元线性微分方程组
\[
\begin{cases}
    \dfrac{7}{5}mr\ddot{y}+\left( \dfrac{6}{5}QBr-\dfrac{2}{5}mr\Omega \right) \dot{x}+\dfrac{1}{5}QrB\Omega y=0\\
    \dfrac{7}{5}mr\ddot{x}-\left( \dfrac{6}{5}QBr-\dfrac{2}{5}mr\Omega \right) \dot{y}+\dfrac{1}{5}QrB\Omega x=0
\end{cases}
\tag{10}
\]
令$\tilde{\xi}=x+\i y$,将(10)式中两式相加,得
\[
\dfrac{7}{5}mr\ddot{\tilde{\xi}}+\left( \dfrac{6}{5}QBr-\dfrac{2}{5}mr\Omega \right)\dot{\tilde{\xi}}\i+\dfrac{1}{5}QB\Omega r\tilde{\xi}=0
\tag{11}
\]
令$\tilde{\xi}=e^{\i\omega t}$,带入(11)式
\[
-\dfrac{7}{5}mr\omega ^{2}-\left( \dfrac{6}{5}QBr -\dfrac{2}{5}mr\Omega \right) \omega +\dfrac{1}{5}QB\Omega r=0
\tag{12}
\]
得
\[
\omega_{1}=\dfrac{3QBr-mr\Omega +\sqrt{\left( 3QBr-mr\Omega \right) ^{2}+7mr^{2}\omega^{2}QB\Omega }}{-7mr}
\tag{13}
\]
\[
\omega_{2}=\dfrac{3QBr-mr\Omega -\sqrt{\left( 3QBr-mr\Omega \right) ^{2}+7mr^{2}\omega^{2}QB\Omega }}{-7mr}
\tag{14}
\]
将两解线性叠加,并重新拆分为$x,y$,有
\[
x=A\cos \left( \omega _{1}t\right) +B\cos\left( \omega_{2}t\right)
\]
\[
y=A\sin \left( \omega _{1}t\right) +B\sin\left( \omega_{2}t\right)
\]
带入初值
\[
\begin{cases}
    A+B=r_0\\
    A\omega _{1}+A\omega _{w}=\Omega r_{0}
\end{cases}
\tag{15}
\]
解得
\[
\begin{cases}
A=-\dfrac{r_{0}\left( \omega _{2}-\Omega \right) }{\omega _{1}-\omega _{2}}\\
B=\dfrac{r_{0}\left( \omega _{1}-\Omega \right) }{\omega _{1}-\omega _{1}}
\end{cases}
\tag{16}
\]
\textbf{评分标准:}\par
共$50$ 分\par
(1)共$5$分 $(2)$各$1$分,$(1),(3)$各$2$分\par
(2)共$45$分 $(4),(6)$各$2$分,$(8),(12),(13),(14),(15),(16)$各$3$分,$(5),(7),(9),(10),(11)$$4$分,指出$\omega_z=0$给$3$分\par
\end{document}