\documentclass{article}
\usepackage{ctex}
\usepackage{amsmath}
\usepackage{graphicx}
\usepackage{wrapfig}
\usepackage{caption}
\usepackage[top=0.8in, bottom=0.8in,left=0.8in, right=0.8in]{geometry}
\usepackage{float} 
\usepackage{subfigure}
\usepackage{subcaption}
\usepackage{bm}
\xeCJKsetup{CJKmath=true} 

\begin{document}
\everymath{\displaystyle}
\section*{手把手系列——黑体辐射(80分)}
本体将手把手教你用最严谨的方法导出Plank黑体辐射公式。
\subsection*{Part A.振荡偶极子场}
推迟势的方程
$$
\begin{cases}
    \vec{A}(\vec{r},t)=\int_{V}^{}\dfrac{\mu_0\vec{j}(\vec{r},t-r/c)}{4\pi r|_{t-r/c}}\mathrm{d}V\\
    \varphi(\vec{r},t)=\int_{V}^{}\dfrac{\rho(\vec{r},t-r/c)}{4\pi \varepsilon_0 r|_{t-r/c}}\mathrm{d}V
\end{cases}
$$
其中$(t-r/c)$表示的是电磁信号从原点传播到场点的时间,这也是推迟势名字的由来。\par
考虑一个振荡的偶极子
\begin{itemize}
\item[(A.1)]求偶极子产生的场。
\end{itemize}

\end{document} 