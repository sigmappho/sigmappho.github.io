\documentclass{article}
\usepackage{ctex}
\usepackage{amsmath}
\usepackage{graphicx}
\usepackage{wrapfig}
\usepackage{caption}
\usepackage[top=0.8in, bottom=0.8in,left=0.8in, right=0.8in]{geometry}
\usepackage{float} 
\usepackage{subfigure}
\usepackage{subcaption}
\usepackage{bm}
\xeCJKsetup{CJKmath=true} 

\begin{document}
\section*{液体表面张力系数的确定(40分)}
众所周知液体有表面张力,但是往往表面张力系数是由实验测定的,下尝试通过理论方式建立。\par
\begin{itemize}
\item[(1)]已知热力学第一定律的微分形式是$\mathrm{d}U=Y\mathrm{d}y+T\mathrm{d}s$,其中$Y$是广义力,$y$是广义坐标,给出表面张力系统的热力学第一定律的微分形式。
\end{itemize}\par
液体内部的分子,其周围所受的力在平均后是各向同性的,但在液体表面,由于上面部分没有液体分子。用“作用力球”来说明,就是液体内部“作用力球”完整,而在表面“作用力球”少了一个球冠,从而导致了其受力并不为零,下给出定量分析的模型。\par
计液体内任意两个相邻的分子之间的相互作用能为$\varepsilon$,且液体内部一个分子与$n$个分子相邻,在表面与$\zeta n$个分子相邻.
\begin{itemize}
\item[(2)]给出内外分子势能的差值。
\item[(3)]设表面的粒子面密度为$\sigma_n$,求形成$\mathrm{d}A_S$面积的表面时做的功,并用$\sigma_n,\varepsilon,n,\zeta$表示表面张力系数$\sigma$。\par
\end{itemize}
\par
下确定$\varepsilon$.考虑液体汽化过程,给定摩尔汽化热$L_m$(认为是从内部分子汽化出去的).
\begin{itemize}
\item[(4)]给出$\varepsilon$,用$L_m,N_A,n$表示。
\item[(5)]认为一个分子占据直径为$d$的空间,给定液体分子的摩尔质量$\mu$和密度$\rho$,用$\zeta,L_m,N_A,\rho,\mu$表示表面张力系数$\sigma$ 
\end{itemize}
\par
现考虑混合液体的表面张力系数,设液体的两种组分为$\mu_1,\rho_1,d_1,n_1,\zeta_1,\sigma_{n_1}$和$\mu_2,\rho_2,d_2,n_2,\zeta_2,\sigma_{n_2}$,其中$n$为数密度,液体内两种组分每个分子均与$x_1,x_2$个分子相邻。两种组分之间的相互作用能为$\varepsilon_{11},\varepsilon_{22},\varepsilon_{12}$,并假设$|\varepsilon_{12}|=\sqrt{|\varepsilon_{11}||\varepsilon_{22}|}$
\begin{itemize}
\item[(6)]给出混合液体的表面张力系数$\sigma_{12}$,用$\mu_1,\rho_1,d_1,n_1,\zeta_1,\sigma_{n_1},x_1,\mu_2,\rho_2,d_2,n_2,\zeta_2,\sigma_{n_2},x_2$
\end{itemize}
\end{document} 