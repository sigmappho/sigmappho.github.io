\documentclass{article}
\usepackage{ctex}
\usepackage{amsmath}
\usepackage{graphicx}
\usepackage{wrapfig}
\usepackage{caption}
\usepackage[top=0.8in, bottom=0.8in,left=0.8in, right=0.8in]{geometry}
\usepackage{float} 
\usepackage{subfigure}
\usepackage{subcaption}
\usepackage{bm}
\xeCJKsetup{CJKmath=true} 
\begin{document}
\section*{简单光学题(40分)(命题:YY)}
\[\]
(1)由费马原理
\[
n x \left( \theta\right) +r\left( \theta\right) =nx_{0}+r_{0}
\tag{1}
\]
由长度约束
\[
x \left( \theta \right) +r\left(\theta\right)  \cos\theta =x_{0}+r_{0}
\tag{2}
\]
$(1)-n(2)$
\[
\left( -1+n\cos \theta \right) r\left( \theta \right) =\left(n-1\right) r_{0}
\tag{3}
\]
可得
\[
r\left( \theta \right) =\dfrac{\left( n-1\right) r_{0}}{-1+n\cos \theta }
\tag{4}
\]
与极坐标下的圆锥曲线标准方程类比
\[
r=\dfrac{p}{1+e\cos \theta }
\tag{5}
\]
得
\[
e=n=\dfrac{c}{a}
\tag{6}
\]
\[
\left( n-1\right) r _{0}=\dfrac{b^{2}}{a}=\left( e^{2}-1\right) a
\tag{7}
\]
\[
a=\dfrac{r_{0}}{n+1}
\tag{8}
\]
\[
c=\dfrac{nr_{0}}{n+1}
\tag{9}
\]
\[
b=\sqrt{\dfrac{n-1}{n+1}}r_{0}
\tag{10}
\]
有
\[
\dfrac{x^{2}}{\left( \dfrac{r_{0}}{n+1}\right) ^{2}}-\dfrac{y^2}{\dfrac{n-1}{n+1}r_0^{2}}=1
\tag{11}
\]
原点在$O$右侧$\dfrac{r_0}{n+1}$
\[\]
(2)
\[r_0\cos\theta +\dfrac{mv}{qB}\sin \theta =f
\tag{12}
\]
\[
y=r\sin \theta 
\tag{13}
\]
\[
\sin \theta =\dfrac{xqB}{mv}
\tag{14}
\]
带入(12)
\[
x+r\sqrt{ {1-\left( \dfrac{xqB}{mv}\right) ^{2}}}=f
\tag{15}
\]
\[
r=\dfrac{f-x}{\sqrt{1-\left( \frac{xqB}{mv}\right) ^{2}}}
\tag{16}
\]
\[
y=\dfrac{x\left( f- x\right) }{\sqrt{\left( \frac{mv}{qB}\right) ^{2}-x^{2}}}
\tag{17}
\]
(3)
利用简谐运动周期与振幅无关的特点,可得电场力是线性恢复力
\[
\alpha =1
\tag{18}
\]
\[
-kyq=m\ddot{y}
\tag{19}
\]
\[
\omega=\sqrt{\dfrac{kq}{m}}
\tag{20}
\]
\[
t=\dfrac{T}{4}=\dfrac{\pi }{2}\sqrt{\dfrac{m}{kq}}
\tag{21}
\]
\[
\left( \dfrac{2f}{\pi v}\right) ^{2}=\dfrac{m}{kq}
\]
\[
k=\dfrac{m}{q}\left( \dfrac{\pi v}{2f}\right) ^{2}
\tag{22}
\]
(4)
类比力学中的莫培督原理与光学的费马原理
\[\delta \int p\cdot dq=0\leftrightarrow \delta \int n\cdot dl=0
\]
取$p\leftrightarrow n,q\leftrightarrow l,m=1$
\[\]
在$y$处进入电场的速度为
\[
v=n(y)
\tag{23}
\]
时间
\[
t=\dfrac{f-x}{n\left( y\right) }=\dfrac{f}{n_{0}}
\tag{24}
\]
\[
E_p=-\dfrac{1}{2}n^{2}
\tag{25}
\]
\[
\omega=\dfrac{\pi n_{0}}{2f}=\sqrt{k}
\tag{26}
\]
\[
F=-ky=-\left(\dfrac{\pi n_{0}}{2f}\right) ^{2}y
\tag{27}
\]
\[
\int F\cdot \mathrm{d}y=\dfrac{1}{2}\left( n^{2}-n(0)^2\right) =-\dfrac{1}{2}\left(\dfrac{\pi n_{0}}{2j}\right) ^{2}y^{2} 
\tag{28}
\]
\[
n=\sqrt{ n_{0}^{2}-\left(\dfrac{\pi n}{2f}\right) ^{2}y^{2}}
\tag{29}
\]
\[
x=f\left[ 1-\sqrt{1-\left( \dfrac{\pi y}{2f}\right) ^{2}}\right]
\tag{30}
\]
\textbf{评分标准:}\par
共$60$ 分\par
(1)共$9$分 $(1),(2),(3),(6),(7),(8),(9),(10),(11)$各$1$分\par
(2)共$8$分 $(12),(13),(14),(15)$各$1$分,$(16),(17)$各$2$分\par
(3)共$11$分 $(19),(20),(21),(22)$各$2$分 $(18)$$3$分\par
(4)共$12$分 $(24),(25),(26),(27),(29),(30)$各$2$分 \par
\end{document}
